\section{Implementation}
    crucial arithmetic
    How it works together
    Reference ideas (Maybe from litterature)
    - Explain how you got th idea
    - Did you get the idea from somewhere else, explain it from where

~10 pages

This section will describe how the reference implemenation from Bernstein was ported, and how the Karatsuba algorithm is implemented in practice.
\subsection{Reference implementation}
The reference implementation was straight forward to port, as the arithmetic itself dosent have to be changed.


The \textit{checksum_compute} function used the \texttt{random()} function given in the C standard libary to fill \texttt{p} and \texttt{n} with random values. In the main implementation \texttit{randombytes()} is used to fill \texttt{p} and \texttt{n} with random values, as \texttit{randombytes()} is supplied with the implementation from the \cite[rainbowgit implemenation]{rainbowgit}, 
As the C standard libary is not available an alternative to the \texttt{random()} function is used. The library \cite[rainbowgit]{rainbowgit} is used for the \texttt{randombytes()} implemenation, this implementation uses \texttt{AES} to give a pseudorandom output. As \textit{AES} isnt supplied either, the \cite[libcrypto library]{libcrypto}, it can be found in \textit{src/libcrypto}. It is maintained by the official RISC-V organization, and is actively being developed.



\subsection{Karatsuba algorithm}