\section{Introduction}
%  2 pages
% Introduce platform, 
% curve / scheme (1 page), 
% background about riscv (2-3 page)\\
% \\
% Forklar noget om PQVexRiscV??? og risc-v??
%  -> The PQVexRiscV\cite{PQVexRiscV} template implements a series of serial-functions. 
A essential part of computing is having a secure cryptographic algorithm. Theese algorithm's has become a crucial part of messaging, and network systems, hence the focus on making these algorithms faster.
The goal of this project is to port Curve25519 to RISC-V, and optimize it for the VexRiscv platform.
\\
Curve25519 is a elliptic cure algorithm released by Daniel J. Bernstein in 2005. The algorithm was originally defined as a Diffie-Hellman function.
ECC uses smaller keys than RSA while providing the same level of security, with faster key generation.
A ECC public key contains a pair of integer coordinates \textit{(x, y)} on the curve. A elliptic point can be compressed to a single X coordinate, which is used by Curve25519.
\\
The Elliptic-curve Diffie-Hellman is a key agreement scheme, is a classical use case which allows two parties to establish a shared secret with 2 different public/private key sets. 
Eliptic-curve Diffie-Hellman (ECDH) is very similar to a classical Diffie Hellman, the difference is that ECDH uses \textit{ECC point multiplication} where the classical approach uses \textit{modular exponentiation}.\\
Karatsuba is an multiplication algorithm which is faster than typical schoolbook multiplication. \\
The ISA RISV-V has been increasing in popularity as it offers a modular and expandable design in a open-source form. The RISC-V implementation VexRiscv is easy to expand, quite modular and optimized for FPGA's.\\
\\
The report and project files is publicly available on the git repository.\\
\href{https://github.com/Clonex/RiscV-ECC}{https://github.com/Clonex/RiscV-ECC}
\pagebreak