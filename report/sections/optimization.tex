\section{Optimization}
~10 pages


-- Skriv om hvor der ELLERS kunne optimeres??


\subsection{Karatsuba}
The Karatsuba algorithm is used to perform multiplication of large numbers with fewer operations, than the school-book multiplication approach.
$n^{\log_{2}3} \approx n^{1.58}$
As multiplication is a 

Multiplication is used intensly during the encryption[] steps, this was a focus of optimization. The reference implementation of the curve uses simple schoolbook multiplication.
Where every single digit is multiplied with the corresponding digit in the other number, and a carry is carried on to the next iteration when the result is bigger than the base. This carry will then be multiplied with the next set of digits.\\
EXAMPLE IMAGE\\
The schoolbook approach requires $n^{2}$ single-digit products. 
This approach is the approach used in the reference implemention, which can be replaced with a more optimal algorithm.
\\

The Karatsuba algorithm 
\label{karat-opti}
\[A_0 \times B_0 + 2^{m}((A_0 + A_1) \times (B_0 + B_1) - A_0 \times B_0 - A_1 \times B_1) + 2^{2m} A_1 \times B_1\]
This approach requires 5 products, with 2 of them being the same multiplications. For this reason 2 of them can be stored and them re-used without any calculations.
\\EXAMPLE IMAGE\\
The Karatsuba algorithm uses more additions compared to the schoolbook approach which uses 4 multiplications and 3 additions, this uses 3 multiplications, 4 additions and 2 subtractions. So the algorithm saves multiplications at the cost of additions and subtractions. Thus is Karatsuba faster if multiplication is more expensive than additions and subtractions. 

