\section{Abstract}
I denne rapport vil jeg beskrive hvordan Karatsuba er blevet brugt til at optimere udregningerne brugt til Curve22519 som er blevet portet over til RISC-V ISA'en. RISC-V er en open source \texttt{open standard instruction set architecture} også kaldet ISA. Det er altså en mere åben tilgang til CPU design.
\\
Karatsuba er en hurtig multiplikations algoritme som kan bruges til store integer værdier.\\En reference implementation af Curve22519 er blevet ported til RISC-V og bliver simuleret på PQVexRiscV CPU'en. Der er så blevet kodet nogle test scripts som bruges til at måle antal cyklusser det tager at køre krypteringen.  
Udfra disse resultater kan der laves en evaluering. Selvom en optimering på $0.676$ megacycles er minimal, er det stadig hurtigere end den originale implementation, samt lægger en god grund til at fortsætte med nogle andre optimeringer.